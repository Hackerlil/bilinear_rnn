\documentclass[a4paper,11pt]{article}
 
\usepackage[english]{babel}
\usepackage[T1]{fontenc}
\usepackage[ansinew]{inputenc}
\usepackage{bbold}
\usepackage{bold-extra}
\usepackage{lmodern}
\usepackage{amsmath}
\usepackage{amsthm}
\usepackage{amsfonts}
\usepackage{gensymb}
\usepackage{mathtools}
\usepackage{subfigure}
\usepackage{theorem}
\usepackage{bm}
\usepackage{xcolor}
\usepackage[unicode]{hyperref}
\usepackage{algorithm}
\usepackage{algorithmic}
\usepackage{tikz}
\usepackage{graphicx}
\usepackage{lipsum}
\usetikzlibrary{positioning}
\usepackage{tikz}
\usepackage{empheq}
\usepackage{booktabs}
\usepackage{authblk}
\usepackage{cals}
%\usepackage{3dplot} %requires 3dplot.sty to be in same directory, or in your LaTeX installation
%\usepackage{amsmath,amsfonts,amssymb,amsthm,epsfig,epstopdf,titling,url,array}
\usepackage{xstring}

%\theoremstyle{definition}
%\newtheorem{defn}{Definition}[section]
%\newtheorem{conj}{Conjecture}[section]
%\newtheorem{exmp}{Example}[section]
\makeatletter

\renewcommand\cals@issue@row{%
\nointerlineskip
\setbox0=\vtop{\hbox to \textwidth{\hskip\leftskip \box\cals@current@cs \hskip\rightskip}}%
  \ht0=0pt \box0
\nointerlineskip
\hbox to \textwidth{\hskip\leftskip\hbox{\cals@issue@rowsep}\hskip\rightskip}%
\nointerlineskip
\hbox to \textwidth{\hskip\leftskip \box\cals@current@row \hskip\rightskip}%
\let\cals@last@rs@below=\cals@current@rs@below
\let\cals@last@context=\cals@current@context}


\newcommand{\change@uppercase@math}{%
  \count@=`\A
  \loop
    \mathcode\count@\count@
    \ifnum\count@<`\Z
    \advance\count@\@ne
  \repeat}

\newcommand{\LSTM}[1]{
  \mathrm{LSTM}(
  %(\begingroup\change@uppercase@math#1\endgroup)
}

\newcommand{\UPDATE}[1]{
  \mathrm{UPDATE}(
  %(\begingroup\change@uppercase@math#1\endgroup)
}

\newcommand{\READ}[1]{
  \mathrm{READ}(
 % (\begingroup\change@uppercase@math#1\endgroup)
}

\newcommand{\ADD}[1]{
  \mathrm{ADD}(
 % (\begingroup\change@uppercase@math#1\endgroup)
}
\newcommand{\MUL}[1]{
  \mathrm{MUL}(
 % (\begingroup\change@uppercase@math#1\endgroup)
}

\newcommand{\CIRC}[1]{
  \mathrm{circ}(
 % (\begingroup\change@uppercase@math#1\endgroup)
}

\newcommand{\ReLU}[1]{
  \mathrm{ReLU}(
 % (\begingroup\change@uppercase@math#1\endgroup)
}


\newcommand{\softmax}[1]{
  \mathrm{softmax}(
 % (\begingroup\change@uppercase@math#1\endgroup)
}


\makeatother

\newcommand*\GetListMember[2]{\StrBetween[#2,\number\numexpr#2+1]{,#1,},,\par}%
\newcommand{\tikzmark}[1]{\tikz[overlay,remember picture] \node (#1) {};}

\def\spvec#1{\left(\vcenter{\halign{\hfil$##$\hfil\cr \spvecA#1;;}}\right)}
\def\spvecA#1;{\if;#1;\else #1\cr \expandafter \spvecA \fi}

\newlength{\MidRadius}
\newcommand*{\CircularSequence}[3]{%
    % #1 = outer circle radius
    % #2 = inner circle radius
    % #3 = seqeunce
    \StrCount{#3}{,}[\NumberOfElements]
    \pgfmathsetmacro{\AngleSep}{360/(\NumberOfElements+1)}
    \pgfmathsetlength{\MidRadius}{(#1+#2)/2}
    \draw [red,  ultra thick] circle (#2);
    \draw [blue, ultra thick] circle (#1);
%    \draw [thick,->] (0, 0) -- (1.0, 0);
%    \draw [thick,->] (0, 0) -- (0.0, 1.0);
    \foreach [count = \Count] \Angle in {0,\AngleSep,..., 360} {%
        \draw [gray, ultra thick] (\Angle:#2) -- (\Angle:#1);
        \pgfmathsetmacro{\MidPoint}{\Angle+\AngleSep/2}
        \node at (\MidPoint:\MidRadius) {\GetListMember{#3}{\Count}};
    }%7
}%


\author{Povilas Daniu\v{s}is}
\author[1]{Povilas Daniu\v{s}is \thanks{povilas.daniusis@gmail.com}}


\title{Bilinear recurrent neural networks (draft)}

\begin{document}
\maketitle
\section{Introduction}

Many sequence learning problems are efficiently solved by recurrent neural networks (RNN's) (e.g. speech recognition \cite{Sak}, machine translation \cite{Sutskever}, image caption generation \cite{Vinyals}). RNN's encode a sequence of inputs into the fixed structure internal state, which is further used to infer a sequence of corresponding outputs. 



Although from theoretical point of view RNN's are capable of performing arbitrary computations \cite{Siegelmann}, in practice training of RNN's is often non-trivial and time consuming process, and therefore new, more efficient architectures and training methods are needed.

This study is devoted to the question of how RNN's can benefit from bilinear products.

Although the inner product representation plays fundamental role  in many machine learning algorithms, they often can benefit from bilinear products as well. For example, bilinear hashing \cite{Gong}, feed forward neural nets \cite{Daniusis}, SVM's \cite{Cai} can be more efficient than the conventional analogues. 


This article contributes to RNN's by applying bilinear products to derive new modification of long short-term memory (LSTM) \cite{Hochreiter} and gated recurrent unit (GRU) \cite{Chung} recurrent neural networks, and conducting an empirical analysis of suggested models. LSTM and GRU were chosen because of their effectiveness.

Main advantages of our approach is that it can be used directly for matrix-valued sequences, is able to capture the structure of such a data more efficiently comparing to conventional analogues, and contains less parameters. Tensorflow \cite{Tensorflow} implementation of suggested bilinear RNN's can be downloaded from \url{https://github.com/povidanius/bilinear_rnn}.




\subsection{Short review of RNN models}
Let $t$ be discrete time variable, and let $x_{t} \in \mathbb{R}^{D_{x}}$ be corresponding input vectors. Simple RNN (SRNN) model is defined by 
recurrence $h_{t} = \phi(Wx_{t} + Uh_{t-1} + b)$, where $h_{t}$ is hidden state vector, 
$W$ and $U$ are parameter matrices, $b$ is bias vector, and $\phi$ - activation function \cite{Elman}. 
Receiving the sequence of inputs $x_{t}$, the model maintains hidden state $h_{t}$, and 
outputs $y_{t} = \psi(Vh_{t} + c)$, where $V$ and $c$ are another parameters, and $\psi$ is output activation function. Although SRNN is able to learn non-trivial sequential regularities, in practice it is rather limited. 


\subsubsection{LSTM}


LSTM \cite{Hochreiter} relies on the combination of two innovations: gate mechanism and additive state update mechanism.

Upon receiving new input $x_{t}$ the LSTM combines it with hidden state into new candidate memory vector.

The input gate controls an integration of this new information into cell's memory, allowing to 
inhibit certain components of candidate cell. Similarly, the forget gate controls integration or previous memory, 
and output gate determines exposition weights of $c_{t}$.

An update of LSTM state $(c_{t}, h_{t})^{T}$ is defined by:
\begin{align}
\begin{split}
\label{LSTM}
i_{t} &=\sigma(W^{i}x_{t} + U^{i}h_{t-1} + b^{i})\\
f_{t} &=\sigma(W^{f}x_{t} + U^{f}h_{t-1} + b^{f})\\
o_{t} &=\sigma(W^{o}x_{t} + U^{o}h_{t-1} + b^{o})\\
\tilde{c}_{t} &= \tanh(W^{c}x_{t} + U^{c}h_{t-1} + b^{c})\\
c_{t} &=f_{t} \bullet c_{t-1} + i_{t} \bullet \tilde{c}_{t}\\
h_{t} &=o_{t} \bullet \tanh(c_{t}), \qedhere
\end{split},
\end{align}

Where $\bullet$ denotes element-wise multiplication, $i_{t}$, $f_{t}$, $o_{t}$ are called input, forget and 
output gates, $\tilde{c}_{t}$ - candidate cell memory, $c_{t}$ - cell memory, and $h_{t}$ - hidden state. 
All aforementioned variables are $D_{h}$- dimensional. Parameter count of LSTM is 
$4 \cdot D_{h} \cdot (D_{x}  + D_{h} + 1)$, and state is defined by $2 \cdot D_{h}$ variables.

The intuition behind LSTM
$c_{t}$ may be interpreted as internal memory of LSTM, while $h_{t}$ - represent its content, 
exposed by the output gate.

\subsubsection{GRU}
Similar and simpler recurrent architecture, gated recurrent unit (GRU) \cite{Chung}, is defined by:

\begin{align}
\begin{split}
\label{LSTM}
u_{t} &=\sigma(W^{i}x_{t} + U^{i}h_{t-1} + b^{i})\\
r_{t} &=\sigma(W^{f}x_{t} + U^{f}h_{t-1} + b^{f})\\
\tilde{h}_{t} &= th(W^{h}x_{t} + r_{t} \bullet (U^{h}h_{t-1}) + b^{h})\\
h_{t} &= u_{t} \bullet \tilde{h}_{t} + (1 - u_{t}) \bullet h_{t-1}, \qedhere
\end{split}
\end{align}

\noindent Reset gate $r_{t}$ controls integration of previous state $h_{t-1}$ into candidate state $\tilde{h}_{t}$, and update gate $u_{t}$ controls integration of candidate state into state update. GRU has $ 3 \cdot D_{h} \cdot (D_{x}  + D_{h} + 1) $ parameters and $D_{h}$-dimensional state variable.




\subsection{Bilinear analogues of LSTM and GRU}


This section describes bilinear LSTM and GRU RNN's. We assume that the inputs $X_{t}$ are $D_{X}^{1} \times D_{X}^{2}$ matrices, and hidden state variables are $D_{H}^{1} \times D_{H}^{2}$ matrices. Bilinear SRNN can now be reformulated as:
$H_{t} = \phi(W_{1}X_{t}W_{2} + U_{1}H_{t-1}U_{2} + B)$, where $\phi$ is non-linear activation function, and parameter matrices of appropriate dimensions.


\subsubsection{Bilinear LSTM (BLSTM)} 
\begin{align}
\begin{split}
I_{t} &=\sigma(W_{1}^{i}X_{t}W_{2}^{i} + U_{1}^{i}H_{t-1}U_{2}^{i}  +  B^{i})\\
F_{t} &=\sigma(W_{1}^{f}X_{t}W_{2}^{f} + U_{1}^{f}H_{t-1}U_{2}^{f}  +  B^{f})\\
O_{t} &=\sigma(W_{1}^{o}X_{t}W_{2}^{o} + U_{o}^{i}H_{t-1}U_{2}^{o}  +  B^{o})\\
\tilde{C}_{t} &=\tanh(W_{1}^{c}X_{t}W_{2}^{c} + U_{1}^{c}H_{t-1}U_{2}^{c}  +  B^{c})\\
C_{t} &= F_{t} \bullet C_{t-1} + I_{t} \bullet \tilde{C}_{t}\\
H_{t} &= O_{t}\bullet \tanh(C_{t}),
\end{split}.
\end{align}



\subsubsection{Bilinear GRU (BGRU)} 
\begin{align}
\begin{split}\nonumber
R_{t} &=\sigma(W_{1}^{i}X_{t}W_{2}^{i} + U_{1}^{i}H_{t-1}U_{2}^{i}  +  B^{i})\\
U_{t} &=\sigma(W_{1}^{f}X_{t}W_{2}^{f} + U_{1}^{f}H_{t-1}U_{2}^{f}  +  B^{f})\\
\tilde{H}_{t} &= th(W_{1}^{c}X_{t}W_{2}^{c} + R_{T} \bullet (U_{1}^{c}H_{t-1}U_{2}^{c})  +  B^{c})\\
H_{t} &= U_{t}\bullet \tilde{H}_{t} + (11^{T} - U_{t})\bullet H_{t-1},
\end{split}.
\end{align}

\subsubsection{Analysis}

The dimension of parameter spaces of BSRNN is $|\theta|_{BSRNN} = D_{H}^{1} (D_{X}^{1} + D_{H}^{1} + D_{H}^{2}) + D_{H}^{2} (D_{X}^{2} + D_{H}^{2})
$, and its state is represented by $D_{H}^{1} D_{H}^{2}$ variables. BLSTM/BGRU contains $4|\theta|_{BSRNN}$ and $3|\theta|_{BGRU}$ parameters respectively.


\section{Computer experiments}

\subsection{Meta-learning of RNN}

We represent parameters of LSTM by $4D_{h} \times (D_{x} + D_{h} + 1)$ matrix.








\begin{thebibliography}{1}



\bibitem{Andrychowicz} Andrychowicz, M.,  Denil, M., Gomez, S., Hoffman, M. W., Pfau, D., Schaul, T., and de Freitas N.  Learning to learn by gradient descent by gradient descent. CoRR, abs/1606.04474, 2016.

\bibitem{Bengio} Bengio,  Y.,  Simard,  P.,  and  Frasconi, P. Learning long-term dependencies with gradient descent is difficult. Neural Networks, IEEE Transactions on, 5(2):157-166, 1994.


\bibitem{Cai} Cai, D., He, X., Han, J. Learning with Tensor Representation, preprint, 2006.

\bibitem{Cheng} Cheng, Y., Yu, F. X., Feris, R.,S., Kumar, S., Choudhary, A., and Chang, S.  An exploration of parameter redundancy in deep networks with circulant projections. In
ICCV, 2015.


\bibitem{Chung} Chung, J., Gulcehre, C., Cho, K., and Bengio, Y. Empirical evaluation of gated recurrent neural networks on sequence modeling. NIPS Deep Learning Workshop, 2014.

\bibitem{Daniusis} Daniusis, P., and Vaitkus, Pr. Neural network with matrix inputs. Informatica, 2008, vol.19 (4): 477-486.


\bibitem{Elman} Elman, J. L. Finding structure in time.  CRL Technical Report 8801, Center for Research in Language, University
of California, San Diego, 1988.

\bibitem{Graves}  Graves, A., Wayne, G., Reynolds, M.,  Harley, T., Danihelka, I., Grabska-Barwi
 nska, A., Colmenarejo, S.G., Grefenstette, E., Ramalho, T., Agapiou, J., et al. Hybrid computing using a neural network with dynamic external memory. Nature, 2016.
 


\bibitem{Gong}  Gong, Y., Kumar, S., Rowley, H. A., and Lazebnik, S. Learning binary codes for high-dimensional data using bilinear projections. CVPR, pages 484-491, 2013.



\bibitem{Haykin} Haykin, S. Neural Networks: A Comprehensive Foundation. 2nd Edition. Prentice Hall, 1998.


\bibitem{Henriques} Henriques,  J. F.,  Caseiro, R.,  Martins, P.,  and Batista, J.   Exploiting the Circulant Structure of Tracking-by-Detection with Kernels.   In ECCV, 2012.


\bibitem{Hochreiter} Hochreiter, S.  and  Schmidhuber J. Long  Short-Term  Memory. Neural  Computation, 9(8): pp. 1735-1780, 1997.



%\bibitem{Kingma} Kingma,  D.,  and  Ba,  J.
%Adam:   A  Method  for Stochastic Optimization. arXiv:1412.6980 [cs.LG], December 2014.


\bibitem{Oppenheim} Oppenheim,  A.  V.,  Schafer, R. W.,  Buck,  J. R. Discrete-time signal processing, volume 5.  Prentice Hall Upper Saddle River, 1999.

\bibitem{Pedregosa} Pedregosa, F., Varoquaux, G.,  Gramfort, A.,  Michel, V., Thirion, B., Grisel, O., Blondel, M., Prettenhofer, P.,  Weiss, R., Dubourg, V.,  Vander-plas, J., Passos, A., Cournapeau, D., Brucher, M., Perrot, M., and Duchesnay, E. Scikit-learn: Machine Learning in Python, Journal of Machine Learning Research, vol. 12, pp. 2825-2830, 2011.


\bibitem{Sak} Sak, H., Senior, A., Rao, K. and Beaufays, F. Fast and Accurate Recurrent Neural Network Acoustic Models for Speech Recognition. In INTERSPEECH, 2015.

\bibitem{Schmidhuber}  Schmidhuber, J. Deep  learning  in  neural  networks:  An  overview. Neural  Networks,  61: 85-117, 2015.

%\bibitem{Srivastava} Srivastava, N., Hinton, G.E., Krizhevsky, A., Sutskever, I., and Salakhutdinov, R.R. Dropout: A simple way to prevent neural networks from overfitting. The Journal of Machine Learning Research, 15(1):1929-1958, 2014.

\bibitem{Siegelmann} Siegelmann, H. T., and Sontag, E. D.  On
the  computational  power  of  neural  nets.
Journal of Computer and System Sciences, 50, 1995.




\bibitem{Sutskever} Sutskever, I., Vinyals, O., and Le, Q. V.. Sequence to sequence learning with neural networks, pp. NIPS 2014.


\bibitem{Tensorflow}  Abadi M., Agarwal A., Barham P., Brevdo E., Chen Z., Citro C., Corrado G., Davis A., Dean J., Devin M., et al. TensorFlow: Large-Scale Machine Learning on Heterogeneous Distributed Systems. 2016. arXiv:1603.04467



\bibitem{Tieleman} Tieleman, T., and Hinton, G.   Lecture 6.5-rmsprop:  Divide the gradient by a running average of its recent magnitude. COURSERA: Neural Networks for Machine Learning, 4, 2012.


\bibitem{Vinyals} Vinyals, O., Bengio, S., Erhan, D. Show and tell: A neural image caption generator. In CVPR, 2015.




\bibitem{Yu} Yu, F.,  Kumar, S., Gong, Y, and Chang, S.,-F.  Circulant binary embedding. In ICML, Beijing, China, 2014, pp. 946-954.

\bibitem{Yu0}  Yu, F.  X.,  Kumar, S.,   Rowley, H.,  and  Chang, S.-F. Compact nonlinear maps and circulant extensions. ArXiv preprint arXiv:1503.03893, 2015.



\bibitem{Zhang} Zhang, M., McCarthy, Z., Finn, C., Levine, S. and Abbeel, P. Learning deep neural network policies with continuous memory states, in IEEE International Conference on Robotics and Automation (ICRA), May 2016, pp. 520-527.




\end{thebibliography}



\end{document}